\chapter{Introduction}
The application of machine learning to seemingly any problem is a constant in current literature spanning many
fields.
Improvements in processing power as well as in the algorithms used for training produced impressive results in image
recognition, protein folding, self-driving cars or robotics.\\
One domain that always lends itself to experimentation are games, since they are an enclosed domain that allow for
good evaluation and clear defined rules, and many real world problems can often be reduced and formulated as games.
\newline
Google's DeepMind showed that reinforcement learning can produce super human play in games such as Go, Chess and
Shogi.[REF]\\
These games can be classed as perfect information games as the complete state of the game is observable.
In the case of Go the possible number of game states,calculated to be around 10\textsuperscript{170} is practically
infinite[ref?], yet self play reinforcement learning algorithms in combination with Monte-Carlo-Tree-Search managed
to achieve super human play, which is a remarkable achievement in
modern times.\\
When the game state is not fully observable we speak of imperfect information games.
Card games, such as Bridge or Skat, are examples of this.
The problem is that exhaustive exploration of future possible states is of little use, since the actual game play is
heavily dependent on the cards that are dealt to each player.\\
Reinforcement learning might offer a way of through the thick fog that is imperfect information.


\section{Motivation}
Imperfect information is present in virtually every real world problem we attempt to solve with computers, so it
seems natural to test proven methods in more complex environments.
We therefore want to explore the limits of current reinforcement learning approaches for imperfect information games, as
we see this as the next achievable milestone in research.
For this we chose the bavarian card game Schafkopf to see if any reasonable playing strength can be achieved using
self-play learning and the state of the art Proximal Policy Optimisation.
It is simple enough to allow us a chance at producing good results, whilst we also have good domain knowledge, which
is beneficial for tackling hard problems such as this.
We will build an environment, baseline agents to play against and attempt at training a reasonable strong agent.


\section{Structure}
In the chapter 2[REF] we will give an overview of other people's work the field of imperfect information games, give
some background on the algorithms used.
In chapter 3 we will explain the rules of Schafkopf and show some basic strategy human players use.
Chapter 4 will explain our approach to problem. TODO