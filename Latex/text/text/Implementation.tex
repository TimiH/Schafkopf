\chapter{Implementation}
In this chapter I will briefly outline the approach we took for the implementation of the Schafkopf environment and
the training pipeline.


\section{Enviroment}
The environment was programmed in Python for its fast development and integration of current Reinforcement libraries
such as PyTorch, and follows the traditional Model-View-Controller(\textbf{MVC}) paradigm.
The \textbf{MVC} is a proven concept for game environments as it cleanly separates the game logic held in the model
with the decision process making of the controller in our case the agents interacting with the environment.
In our implementation there are two main components necessary to play a hand of Schafkopf:
\begin{description}
    \item[Game Object] Acts as the model and handles the game logic, game phases and action validation
    \item[Player Object] Acts as the controller and makes an agents individual decision based on the implemented logic
\end{description}

\subsection{Game Object}
The game object is initialised with a list of four agent objects, a lead variable that determines the player position
with the \textbf{Lead} role.
Optionally a seed variable can be passed to control
Internally it follows the game phases defined in \ref{gamephases} .\\
In the\textbf{ Setup phase} it creates a shuffled 32 card deck and deals each player eight cards.
The deck's order can optionally be controlled via a seed variable, which can be useful for debugging or replaying
certain games for evaluation, for example if we want only seeds that give players \textbf{Solo} cards.
\newline
During \textbf{Bidding phase} each player in their respective order at the virtual table is given their hand in
combination with their valid bidding options, which depend on previous bids as well as their hand, by calling the
players' bidding methods.
Once every player has returned their bid the \textbf{Bidding phase} concludes, the highest bid and resulting game
mode can be
determined.
If at this stage no contract has been decided the hand ends immediately.
The hand moves now into the \textbf{Trick phase} after variables such as team composition, run away possibilities and
other game mode induced variables have been evaluated.
\newline
In \textbf{Trick phase} for each of the eight tricks we perform the same loop of actions starting with the leading
player:
\begin{enumerate}
    \item Set the players current hand in the corresponding player object
    \item Determine the valid cards the player may play
    \item Pass the game state,current trick history and valid cards to player by calling player's \textit{playCard}
    method
    \item Check if the action returned by the player is valid
\end{enumerate}
After all players played their card, the trick winner is determined, the point scores are updated, and played cards
are removed from the corresponding hands.
After \textbf{Team Mode} it is also important to check if players \textbf{ran away}} or \textbf{searched} as this is
vital to allow for correct deduction of valid cards that can be played.
\newline
Once all eight tricks have been played successfully the \textbf{Scoring phase} commences and a winner and resulting
reward is calculated and the hand is concluded.

[pseudocode]
\subsection{Game State}
The game state holds all information that define a hand at any moment and is held and passed as dictionary throughout
the environment.
\newline
\begin{table}[h!]
\begin{tabular}{lll}
Name              & Type           & Description                                                             \\
\hline
Hands             & List of Cards  & Description                                                             \\
Game Mode         & Category       & Can be one of eigth possible contracts                                  \\
Lead              & Int            & Table position of leading player                                        \\
Scores            & List of Int    & List of length four                                                     \\
Trick History     & Tuple of Cards & Tuples that keep track of pass tricks                                   \\
Cards Played      & List of Cards  & List of all the cards played (Not necessary, but useful for perfomance) \\
Seed              & Int            & Records the Seed for debugging and evaluation                           \\
Offensive Players & Int            & List of table positions of bid winner and partner                       \\
Run Away Possible & Bool           & Can partner technically run away                                        \\
Ran Away          & Bool           & Did partner run away                                                    \\
Searched          & Bool           & Has been searched
\end{tabular}
\caption{Game state and its members.}
\label{tab:gamestate}
\end{table}

\subsection{Player Class}
The player class is a template class \ref{lst:playerclass} that every agent inherits from.
It handles an agent's decision process during the bidding phase as well as the trick phase.
The two main methods that any agent has to shadow are \textit{makebid()} and \textit{playCard()}.
These are called by the environment and make the agent the controller in the \textbf{MVC} paradigm.
\newline
\begin{lstlisting}[!H][language=Python,label={lst:playerclass}]
import random
class Player():
def __init__(self, name):
self.name = name
self.hand = []
self.position = None

def setHand(self, cards):
self.hand = []
self.hand = cards

def setPosition(self, postion):
self.position = postion

def makeBid(self, validBids):
return random.choice(validBids)

def playCard(self, validCards, gameState, trickHistory):
return random.choice(validCards)
\end{lstlisting}
\section{Reinforcment Learning Setup}
\subsection{PPO}
\subsection{One-Hot-Encoding}
\subsection{Py}